\documentclass[12pt, oneside, letterpaper]{notes}

\begin{document}
\title{Lecture 5 Notes for STT861}
\author{Kenyon Cavender}
\date{2019-09-09}
\maketitle

\section{Review}
$\textbf{S} (\neq \emptyset)$ Sample Space
$\mathscr{A} = \sigma$ - field
A probability $P: \mathscr{A} \mapsto \mathbb{R}$ satisfies
	\index $P(A) \geq 0 \: \forall A \in \mathscr{A} $\\
	\index $P(\textbf{S}) = 1$ \\
	\index If $A_n \in \mathscr{A}, n=1, 2, ...$ and $A_n's$ are parwise disjoint
	then $P(\cup_{n=1}^{\infty} A_n) = \sum_{n=1}^{\infty} P(A_n)$

\section{Class Notes}
\begin{myex}
	Let $\textbf{S} = \{s_1, s_2, ...\}$ be a countable sample  space \\
	Let $\mathscr{A} = \mathscr{P}(\textbf{S})$
	Suppose $\{P_n : n \geq 1\} $ in a sequence satisfying:
	\begin{enumerate}
		\item $p_n \geq 0$ 
		\item $\sum_{n=1}^{\infty}p_n = 1$
	\end{enumerate}
	For any $A \in \mathscr{A}$ define: $P(A) = \sum_{\{j: s_j \in A\}} p_j $

\begin{remark}
	Suppose: \\
	$A = \{s_3, s_{10}, s_{19} \}$ \\
	$P(A) = p_3 + p_{10} + p_{19}$ \\
	\vspace{3pt}
	
	\noindent Suppose: \\	
	$A_1 = \{s_{11}, s_{12}, s_{13} ... \}$ \\
	$A_2 = \{s_{21}, s_{22}, s_{23} ... \}$ \\
	$A_3 = \{s_{31}, s_{32}, s_{33} ... \}$ etc.\\
	All $A_n$ are pairwise disjoint and therefore, $s_{ij}$ are distinct
	for all $i,j$ \\
	$P(\cup_{n=1}^{\infty}A_n) = P(\cup_{n-1}^{\infty}\{s_{n1}, s_{n2}...\} )$ \\
	\indent $ = P(\cup_{n=1}^{\infty} \cup_{j=1}^{\infty} \{s_{nj}\}) =
	\sum_{n=1}^{\infty}(\sum_{j=1}^{\infty} p_{nj})  $ \\
	As $A_n = \{s_{n1}, s_{n2} ... \}$ therefore $P(A_n) = 
	\sum_{j=1}^{\infty}p_{nj} $ \\
	\vspace{3pt}

	\noindent Suppose: \\
	$\textbf{S} = \{1,2,3,4 ...\}$ and $p_n = (frac{1}{2})^n n=1,2,3 ...$ \\
	Then, $\sum_{n-1}^{\infty} = 1$ by geometric sum:
	$\sum_{n=0}^{\infty}a^n = \frac{1}{1-a}$ if $|a| < 1 $ \\
	Therefore, $$\sum_{n=1}^{\infty}(\frac{1}{2})^n 
	= \sum_{n=0}^{\infty}(\frac{1}{2})^n-1 = \frac{1}{1-1/2} -1 = 1$$

\end{remark}

\end{myex}










\end{document}

























