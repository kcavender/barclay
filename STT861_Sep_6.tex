\documentclass[14pt, oneside, letterpaper]{notes}

\begin{document}
\title{Class Notes for STT861}
\author{Kenyon Cavender}
\date{2019-09-06}
\maketitle

\section{Review}
\subsection*{Definitions}
$\textbf{S} ( \neq \emptyset )$ is the sample space. \\
$\mathscr{A} :  \alpha  $ - field on \textbf{S} \\
A set function $P: \mathscr{A} \mapsto \mathbb{R}$ is a probability if it satisfied \\
\indent i) $P(A) \geq 0 \: \: \forall \: A \in \mathscr{A}$ \\
\indent ii) $P(S) = 1 $\\
\indent iii) if $A_1 , A_2 , ... \in \mathscr{A}$ are pairwise disjoint, then \\
\indent \indent $ P(\cup_{i = 1}^{\infty} A_i ) = \sum_{i=1}^{infty} P( A_i )$

\subsection*{Consequences}
\begin{enumerate}
	\item $P(\emptyset) = 0$ 
	
	\item if $A$ and $B$ are pairwise disjoint, then $P(A \cup B) = P(A) + P(B)$ \\
	General Form: if $A_1$, ... , $A_n$ are pairwise disjoint, then \\
	$P(\cup_{i=1}^n A_i) = \sum_{i=1}^n P(A_i)$ 

	\item $P(A^c) = 1-P(A)$

	\item if $A \subset B$, then $P(A) \geq P(B)$ and $P(B \setminus A) = P(B) - P(A)$ 

	\item $P(A) \geq 1,  \: \forall \: A \subset \mathscr(A)$ 
\end{enumerate}

\section{Class Notes}
\begin{mydef}
	A collection of sets $\{E_1, E_2, ...\}$ is called a \textbf{partition} 
		of event A if: \\
	\indent i) $E_i \cap E_j = \emptyset, \: \forall \: i \neq j$ 
		\textit{(pairwise disjoint)} \\
  	\indent ii) $\cup_{i=1}^{\infty}E_i = A$ 
		\textit{(exhaustive)}
\end{mydef}

\begin{remark}
	Partition $\{E_1, \: ..., \: E_n \}$ is a finite partition
\end{remark}

\begin{remark}
	For \textbf{S}, $\{A, A^c\}$ is a partition
\end{remark}

\begin{remark}
	If $E_n: n \geq 1$ is a partition of $A$, then $P(A) = \sum_{i=1}^{\infty} P(E_i)$
\end{remark}

\subsection*{More Consequences} 
\begin{enumerate}[resume]
	\item Suppose $A$ and $B$ are two events.  Then $\{A \cap B, A^c \cap B\}$
	is a partition of $B$.  Also, $P(A^c \cap B) = P(B) - P(A \cap B)$ 

	\begin{myproof}
		$(A \cap B) \cap (A^c \cap B) = 
		(A \cap A^c) \cap (B \cap B) = 0$  \textit{(pairwise disjoint)} \\
		$(A \cap B) \cup (A^c \cap B) = (A \cup A^c) \cap B = B $ 
		\textit{(exhaustive)} 
	\end{myproof}

	\item $A$ and $B$ are two events.  $P(A \cup B) = P(A) + P(B) - P(A \cap B)$ \\
	(This is the generalized version of b)) 

	\item if $\{C_1, C_2, ... \}$ is a partition of \textbf{S}, then \\
	$P(A) = \sum_{i=1}^{\infty} P(A \cap C_i)$ 

	\item\textbf{Boole's Inequality} \\
	\indent $P(\cup_{i=1}^{\infty} A_i) \leq \sum_{i=1}^{\infty}P(A_i)$ 
	
	\item \textbf{Bonferroni's Inequality}\\
	$P(\cap_{i=1}^{\infty} A_i) \geq 1 - \sum_{i=1}^{\infty} P(A_i^c)$
\end{enumerate}

\begin{remark}
	Proving Bonferroni from Boole: \\
	\indent $P(\cup_{i=1}^{\infty} A_i^c) \leq 
	\sum_{i=1}^{\infty} P(A_i^c)$ \\
	\indent $1- P(\cup_{i=1}^{\infty} A_i^c) \:  \geq
	1 - \sum_{i=1}^{\infty} P(A_i^c)$ \\
	\indent $ = P[(\cup_{i=1}^{\infty} A_i^c)^c] \: \geq$ ...  \\
	\indent $ = P[\cap_{i=1}^{\infty} (A_i^c)^c] \: \geq$ ... \\
	\indent $ = P(\cap_{i=1}^{\infty} A_i ) \: \geq
	1- \sum_{i=1}^{\infty} P(A_i^c)$ 
\end{remark}

\begin{mydef}
	A sequence of events $\{A_1, A_2 ...\}$ is \textbf{increasing to event $A$} if: \\
	\indent $A_1 \subset A_2 \subset ...$  \\
	\indent and $A = \cup_{n=1}^{\infty} A_n$ \\
	\indent Notation: $A_n \uparrow A$
\end{mydef}

\begin{mydef}
	Similarly, $B_n \downarrow B$ if $B_1 \supset B_2 \supset ... $ and
	$B = \cap_{n=1}^{\infty} B_n$ 
\end{mydef}

\begin{enumerate}[resume]	
	\item If $A_n \uparrow A$, then $P(A) = \lim_{n \to \infty} P(A_n)$ 

	\item If $B_n \downarrow B$ then $P(B) = \lim_{n \to \infty} P(B_n)$ 
\end{enumerate}













\end{document}























