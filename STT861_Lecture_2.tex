\documentclass[12pt, oneside, letterpaper]{notes}

\begin{document}
\title{Lecture 2 Notes for STT861}
\author{Kenyon Cavender}
\date{2019-08-30}
\maketitle

\section{Review}

\begin{mydef}
  A \textbf{random experiment} is an action which will result 
  in one of the many possible outcomes. 
\end{mydef}

\begin{mydef}
  A \textbf{sample space} is the collection of all possible 
  outcomes of a random experiment.  We shall denote by \textbf{S}.
\end{mydef}

\begin{mydef}
  An \textbf{Event} is a subset of sample space \textbf{S} for 
  which we can define probability.
\end{mydef}

\section{Class Notes}
\begin{mydef}
  \textbf{Intersection} A, B are two events.  \\
  \indent $A \cap B = \{x: x \in A$ and $x \in B\}$
\end{mydef}

\begin{mydef}
  \textbf{Union} A, B are two events. \\
  \indent $A \cup B = \{x: x \in A$ or $x \in B$ or $both\}$
\end{mydef}

\begin{mydef}
  A and B are \textbf{disjoint} if $A \cap B = \emptyset$
\end{mydef}

\noindent \underline{\textbf{Properties of set theory}}  \\
\textbf{Commutative}  \\
\indent $A \cup B = B \cup A$ and $A \cap B = B \cap A$ \\
\textbf{Associative}\\
\indent $(A \cup B) \cup C = A \cup (B \cup C) =: A \cup B \cup C$ \\
\indent $(A \cap B) \cap C = A \cap (B \cap C) =: A \cap B \cap C$ \\
\textbf{Distributive}\\
\indent $(A \cup B) \cap C = (A \cap C) \cup (B \cap C)$ \\
\indent $(A \cap B) \cup C = (A \cup C) \cap (B \cup C)$ \\
\textbf{De Morgan's Law}\\
\indent $(A \cup B)^c = A^c \cap B^c$ \\
\indent $(A \cap B)^c = A^c \cup B^c$ \\

\begin{remark}
	Proving Distributive Property: \\
	$(A \cup B) \cap C = (A \cap C) \cup (B \cap C)$ \\
	
	\begin{myproof}
		Prove left direction: \\
		$x \in (A \cup B) \cap C \Rightarrow x \in A \cup B$ and $x \in C$ \\
		$\Rightarrow (x \in A$ or $x \in B)$ and $x \in C$ \\
		$\Rightarrow (x \in A$ and $x \in C)$ 
		or $(x \in A$ and $x \in C)$ or both \\
		$\Rightarrow  x \in A \cap C$ or $x \in B \cap C$ or both \\
		$\Rightarrow  x \in (A\cap C) \cup (B \cap C)$ \\
		Right direction is reverse of above.
	\end{myproof}
\end{remark}

\begin{mydef}
  \textbf{Set Difference}\\
  \indent $A \setminus B = \{x: x \in A$, but $x \notin B \}$
\end{mydef}

\begin{mydef}
  \textbf{Symmetric Difference}\\
  \indent $A \triangle B = \{x: x \in A \setminus B$, or 
  $x \in B \setminus A \}$
\end{mydef}

\begin{mydef}
  A set $A$ is \textbf{finite} if there exists a 1-1 fn $A \mapsto 
  \{ 1, 2, ..., n \}$ for some $n \in \mathbb{N}$
\end{mydef}

\begin{mydef}
	A set $A$ is \textbf{countably infinite} if there exists a one-to-one
	function from $A \mapsto \mathbb{N}$.
\end{mydef}

\begin{mydef}
	A set is called \textbf{coutable} if it is either finite or countably 
	infinite.
\end{mydef}

\subsection*{Consequences}
\begin{enumerate}
	\item $A \cap B \subset A$ and $A \cap B \subset B$
	
	\item $A \cap A = A$

	\item if $A \subset B $ then $A \cap B = A $
	
	\item $A \subset A \cup B$ and $B \subset A \cup B$

	\item $A \cup A = A$

	\item if $ A \subset B$ then $A \cup B = B$

	\item $A \cup A^c = \textbf{S}$ and $A \cap A^c = \emptyset$

	\item if $A \subset C$ and $B \subset C$ 
	then $A \cap B \subset A \cup B \subset C$

	\item $\emptyset$ is disjoint to all events

	\item if $a \cap B = \emptyset$ then $A \subset B^c$
	and $B \subset A^c$

	\item if $A \subset B$, then $B^c \subset A^c$
	
	\item $A \setminus B = A \cap B^c$ and $B \setminus A = A^c \cap B$
	
	\item $A \triangle B = (A \cup B) \setminus (A \cap B) $
\end{enumerate}

\begin{remark}
	There is some stuff here about lambda that I don't fully grok
%	Let $A_1, A_2, ...$ be a sequence of sets, then 
%	$A_1 \cup A_2 \cup ... = \cup_{k=1}^{\infty}A_k
%	= \{x: x \in A_i$ for some $i \in \mathbb{N}$ \\
	
\end{remark}













\end{document}

























