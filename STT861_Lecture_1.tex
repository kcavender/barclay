\documentclass[12pt, oneside, letterpaper]{notes}

\begin{document}
\title{Lecture 1 Notes for STT861}
\author{Kenyon Cavender}
\date{2019-08-28}
\maketitle

\section{Lecture One}

\begin{myex}
  Toss a fair coin; what is the probability of obtaining heads? 
  $P(H) = \frac{1}{2}$
\end{myex}

\begin{myex}
  Throw a fair die; what is the probability of obtaining 6?  
  What about obtaining an even number? \\
\indent$P(6) = \frac{1}{6}$ \\
\indent$P(n = 2,4,6) = \frac{1}{2}$ 
\end{myex}

\noindent The probability is not the realized result, but the 
  convergence of results as the number of iterations approaches 
  infinity.  Review the Weak Law of Large Numbers.

\begin{mydef}
  A \textbf{random experiment} is an action which will result 
  in one of the many possible outcomes. 
\end{mydef}

\begin{mydef}
  A \textbf{sample space} is the collection of all possible 
  outcomes of a random experiment.  We shall denote by \textbf{S}.
\end{mydef}

\begin{mydef}
  A \textbf{set} is a collection of some \underline{well defined} 
  objects.
\end{mydef}

\begin{mydef}
  \textbf{Outcomes} are also called \textbf{sample points}.
\end{mydef}

\begin{mydef}
  An \textbf{Event} is a subset of sample space \textbf{S} for 
  which we can define probability.
\end{mydef}

\begin{mydef}
  Suppose A and B are two sets.  $A \subset B$ (A is a 
  \textbf{subset} of B) if $x \in A$ implies $x \in B$. 
  If $A \subset B$ and $B \subset A$ then $A=B$.
\end{mydef}

\begin{mydef}
  A set is called an empty set (or \textbf{null set}) if it contains 
  no elements.\\
  \indent Notation: $\{ \emptyset \}$ \\
  \indent Convention: $\emptyset \subset A$, for any set A \\
  \indent Corrolary: $\forall A, \: \emptyset \subset A \subset \textbf{S}$
\end{mydef}

\begin{mydef}
  \textbf{Complement} $A^c$ is the set such that $x \in A^c 
  \Rightarrow x \notin A$.\\
  \indent In other words, $A^c = \{ x : x \notin A \}$ \\
  \indent Notation: $A^c \: or \: A' \: or \: \overline{A}$
\end{mydef}

\subsection*{Results}
i) \indent $(A^c)^c = A$ \\
ii) \indent $\textbf{S}^c = \emptyset $\\
iii) \indent $ \emptyset^c = \textbf{S} $\\
iv) \indent if $A \subset B$, then $B^c \subset A^c$


\begin{mydef}
  \textbf{Intersection} A, B are two events.  \\
  \indent $A \cap B = \{x: x \in A and x \in B\}$
\end{mydef}

\begin{mydef}
  \textbf{Union} A, B are two events. \\
  \indent $A \cup B = \{x: x \in A or x \in B or both\}$
\end{mydef}

\begin{mydef}
  A and B are \textbf{disjoint} if $A \cap B = \emptyset$
\end{mydef}

\noindent \underline{\textbf{Properties of set theory}}  \\
\textbf{Commutative}  \\
\indent $A \cup B = B \cup A$ and $A \cap B = B \cap A$ \\
\textbf{Associative}\\
\indent $(A \cup B) \cup C = A \cup (B \cup C) =: A \cup B \cup C$ \\
\indent $(A \cap B) \cap C = A \cap (B \cap C) =: A \cap B \cap C$ \\
\textbf{Distributive}\\
\indent $(A \cup B) \cap C = (A \cap C) \cup (B \cap C)$ \\
\indent $(A \cap B) \cup C = (A \cup C) \cap (B \cup C)$ \\
\textbf{De Morgan's Law}\\
\indent $(A \cup B)^c = A^c \cap B^c$ \\
\indent $(A \cap B)^c = A^c \cup B^c$ \\

\begin{mydef}
  \textbf{Set Difference}\\
  \indent $A \setminus B = \{x: x \in A$, but $x \notin B \}$
\end{mydef}

\begin{mydef}
  \textbf{Symmetric Difference}\\
  \indent $A \triangle B = \{x: x \in A \setminus B$, or 
  $x \in B \setminus A \}$
\end{mydef}

\begin{mydef}
  A set A is \textbf{finite} if there exists a 1-1 fn $A \mapsto 
  \{ 1, 2, ..., n \}$ for some $n \in \mathbb{N}$
\end{mydef}

\begin{mydef}
  A set with only one outcome from \textbf{S} is called a \textbf{simple event}.
\end{mydef}

\begin{mydef}
  A set with more than one outcome is known as a \textbf{composite event}.
\end{mydef}

\begin{mydef}
  A \textbf{set function} is a function defined on a set.
\end{mydef}

\begin{mydef}
  \textbf{Probability} is a set function which takes a real value.  
\end{mydef}












\end{document}

























