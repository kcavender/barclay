\documentclass[14pt, oneside, letterpaper]{notes}
\usepackage{mathtools}

\begin{document}
\title{STT 861 Compendium - Part Two}
\author{Kenyon Cavender}
\maketitle

\section*{Expected Values}

\begin{mydef}
	The \textbf{expected value} of a random variable $g(X)$: \\
	\[
	\mathbb{E}(g(X)) = 
		\begin{dcases}
			\int_{-\infty}^{\infty} g(x)f_X(x)dx 
			& \text{if $X$ is continuous} \\			
			\sum_{X}g(x)f_X(x)
			& \text{if $X$ is discrete}
		\end{dcases}
	\]
\end{mydef}

\begin{remark}
If $-\infty \leq \mathbb{E}(g(x)) \leq \infty$ we say the
expectation of $g(x)$ exists.  Else it does not exist. \\
\indent notation: $\left| \mathbb{E}(g(x)) \right| 
\leq \infty$ \\
\indent In particular, if $g(x) = x$, then we get the expected
value of $X$: \\
	\[
	\mathbb{E}(g(X)) = 
		\begin{dcases}
			\int xf_X(x)dx 
			& \text{if $X$ is continuous} \\			
			\sum xf_X(x)
			& \text{if $X$ is discrete}
		\end{dcases}
	\]
\indent This is called the \textbf{mean} of r.v. $X$ \\
\indent Also denoted by $\mu$ or $\mu_X$
\end{remark}

\begin{remark} \textbf{Theorem} \\
\begin{enumerate}
	\item $\mathbb{E}(ag(x) + b) = a\mathbb{E}(g(x)) + b
		\text{$a$,$b$ real constants}$
	\item $\text{If} g(x) \geq 0 \text{for all} x \in 
		\mathbb{R} \text{, then} \mathbb{E}(g(x)) \geq 0$
	\item $\text{If} g_1(x) \geq g_2(x) \text{for all} 
		x \in \mathbb{R} \text{, then} 
		\mathbb{E}(g_1(x)) \geq \mathbb{E}(g_2(x))$
	\item For any real constants $a$,$b$ if $a \leq X 
		\leq b$ then $a \leq \mathbb{E}(X) \leq b$

\end{enumerate}
\end{remark}



\end{document}























