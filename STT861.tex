\documentclass[12pt, oneside, letterpaper]{notes}

\begin{document}
\title{STT 861 Compendium}
\author{Kenyon Cavender}
\maketitle

\section*{Definitions}

\begin{mydef}
  A \textbf{random experiment} is an action which will result 
  in one of the many possible outcomes. 
\end{mydef}

\begin{mydef}
  A \textbf{sample space} is the collection of all possible 
  outcomes of a random experiment.  We shall denote by \textbf{S}.
\end{mydef}

\begin{mydef}
  An \textbf{Event} is a subset of sample space \textbf{S} for 
  which we can define probability.
\end{mydef}

\begin{mydef}
  Suppose A and B are two sets.  $A \subset B$ (A is a 
  \textbf{subset} of B) if $x \in A$ implies $x \in B$. 
  If $A \subset B$ and $B \subset A$ then $A=B$.
\end{mydef}

\begin{mydef}
  A set is called an empty set (or \textbf{null set}) if it contains 
  no elements.\\
  \indent Notation: $\{ \emptyset \}$ \\
  \indent Convention: $\emptyset \subset A$, for any set A \\
  \indent Corrolary: $\forall A, \: \emptyset \subset A \subset \textbf{S}$
\end{mydef}

\begin{mydef}
  \textbf{Complement} $A^c$ is the set such that $x \in A^c 
  \Rightarrow x \notin A$.\\
  \indent In other words, $A^c = \{ x : x \notin A \}$ \\
  \indent Notation: $A^c$ or $A'$ or $\overline{A}$
\end{mydef}

\begin{mydef}
  \textbf{Intersection} A, B are two events.  \\
  \indent $A \cap B = \{x: x \in A$ and $x \in B\}$
\end{mydef}

\begin{mydef}
  \textbf{Union} A, B are two events. \\
  \indent $A \cup B = \{x: x \in A$ or $x \in B$ or both$\}$
\end{mydef}

\begin{mydef}
  A and B are \textbf{disjoint} if $A \cap B = \emptyset$
\end{mydef}

%work on fixing this formatting
\noindent \underline{\textbf{Properties of set theory}}  \\
\textbf{Commutative}  \\
\indent $A \cup B = B \cup A$ and $A \cap B = B \cap A$ \\
\textbf{Associative}\\
\indent $(A \cup B) \cup C = A \cup (B \cup C) =: A \cup B \cup C$ \\
\indent $(A \cap B) \cap C = A \cap (B \cap C) =: A \cap B \cap C$ \\
\textbf{Distributive}\\
\indent $(A \cup B) \cap C = (A \cap C) \cup (B \cap C)$ \\
\indent $(A \cap B) \cup C = (A \cup C) \cap (B \cup C)$ \\
\textbf{De Morgan's Law}\\
\indent $(A \cup B)^c = A^c \cap B^c$ \\
\indent $(A \cap B)^c = A^c \cup B^c$ 

\begin{mydef}
  \textbf{Set Difference}\\
  \indent $A \setminus B = \{x: x \in A$, but $x \notin B \}$
\end{mydef}

\begin{mydef}
  \textbf{Symmetric Difference}\\
  \indent $A \triangle B = \{x: x \in A \setminus B$, or 
  $x \in B \setminus A \}$
\end{mydef}

\begin{mydef}
  A set A is \textbf{finite} if there exists a 1-1 fn $A \mapsto 
  \{ 1, 2, ..., n \}$ for some $n \in \mathbb{N}$
\end{mydef}

%
%End of Lecture 1
%

\begin{mydef}
  A set $A$ is \textbf{finite} if there exists a 1-1 fn $A \mapsto 
  \{ 1, 2, ..., n \}$ for some $n \in \mathbb{N}$
\end{mydef}

\begin{mydef}
	A set $A$ is \textbf{countably infinite} if there exists a one-to-one
	function from $A \mapsto \mathbb{N}$.
\end{mydef}

\begin{mydef}
	A set is called \textbf{coutable} if it is either finite or countably 
	infinite.
\end{mydef}

%
%End of lecture 2
%

\begin{mydef}
	$\mathscr{A}$ is a collection of subsets of $\textbf{S}[\neq \emptyset]$
	satisfying: \\
	\indent i) $\textbf{S} \in \mathscr{A}$ \\
	\indent ii) if $ A \subset \mathscr{A}$, then $A^c \in \mathscr{A}$ \\
	\indent iii) if $A_1, A_2, ... \in \mathscr{A}$ then 
	$\cup_{i=1}^{\infty} A_i \in \mathscr{A} $ \\
	We call $\mathscr{A}$ a $\sigma$ - algebra (or $\sigma$ - field) \\
	Any domain should be a $\sigma$ - field
\end{mydef}

\begin{mydef}
	$(\textbf{S}, \mathscr{A}, P)$ is a (probability) measure space
\end{mydef}

\begin{mydef}
	Given sample space $\textbf{S} (\neq \emptyset)$, and the 
	measurable space $(\textbf{S}, \mathscr{A})$ \\
	A function $P: \mathscr{A} \mapsto \mathbb{R}$ is called 
	probability if it satisfies:
	\begin{enumerate}
		\item $P(A) \geq 0$ for any $A \in mathscr{A}$
		\item $P(\textbf{S}) = 1$
		\item if $A_1, A_2 ...$ are disjoint sets from $\mathscr{A}$,
		then $P(\cup_{i=1}^{\infty} A_i) = \sum_{i=1}^{\infty} P(A_i)$
	\end{enumerate}
\end{mydef}

%
%End Lecture 3
%

\begin{mydef}
	A collection of sets $\{E_1, E_2, ...\}$ is called a \textbf{partition} 
		of event A if: \\
	\indent i) $E_i \cap E_j = \emptyset, \: \forall \: i \neq j$ 
		\textit{(pairwise disjoint)} \\
  	\indent ii) $\cup_{i=1}^{\infty}E_i = A$ 
		\textit{(exhaustive)}
\end{mydef}

\begin{mydef}
	A sequence of events $\{A_1, A_2 ...\}$ is \textbf{increasing to event $A$} if: \\
	\indent $A_1 \subset A_2 \subset ...$  \\
	\indent and $A = \cup_{n=1}^{\infty} A_n$ \\
	\indent Notation: $A_n \uparrow A$
\end{mydef}

\begin{mydef}
	Similarly, $B_n \downarrow B$ if $B_1 \supset B_2 \supset ... $ and
	$B = \cap_{n=1}^{\infty} B_n$ 
\end{mydef}

%
%End Lecture 4
%
\begin{mydef}
Counting Methods
\begin{table}[h!]
	\begin{tabular}{r|c|c}
		& \textbf{WOR} & \textbf{WR} \\
		\hline
		
		\textbf{Ordered} 
		& $\displaystyle \frac{n!}{(n-r)!} $ 
		& $\displaystyle n^r $ \\
		
		\textbf{Unordered} 
		& $\displaystyle \frac{n!}{r!(n-r)!} 
		= {n \choose r} $ 
		& $\displaystyle {n+r-1 \choose r} $ \\
	\end{tabular}
\end{table}

\end{mydef}

%
%End Lecture 5
%













\section*{Results}
\begin{enumerate}
\item $(A^c)^c = A$
\item $\textbf{S}^c = \emptyset $
\item $ \emptyset^c = \textbf{S} $
\item if $A \subset B$, then $B^c \subset A^c$
\end{enumerate}

%
%End of Lecture 1
%

\begin{remark}
	Proving Distributive Property: \\
	$(A \cup B) \cap C = (A \cap C) \cup (B \cap C)$ \\
	
	\begin{myproof}
		Prove left direction: \\
		$x \in (A \cup B) \cap C \Rightarrow x \in A \cup B$ and $x \in C$ \\
		$\Rightarrow (x \in A$ or $x \in B)$ and $x \in C$ \\
		$\Rightarrow (x \in A$ and $x \in C)$ 
		or $(x \in A$ and $x \in C)$ or both \\
		$\Rightarrow  x \in A \cap C$ or $x \in B \cap C$ or both \\
		$\Rightarrow  x \in (A\cap C) \cup (B \cap C)$ \\
		Right direction is reverse of above.
	\end{myproof}
\end{remark}

\begin{enumerate}
	\item $A \cap B \subset A$ and $A \cap B \subset B$
	\item $A \cap A = A$
	\item if $A \subset B $ then $A \cap B = A $
	\item $A \subset A \cup B$ and $B \subset A \cup B$
	\item $A \cup A = A$
	\item if $ A \subset B$ then $A \cup B = B$
	\item $A \cup A^c = \textbf{S}$ and $A \cap A^c = \emptyset$
	\item if $A \subset C$ and $B \subset C$ 
	then $A \cap B \subset A \cup B \subset C$
	\item $\emptyset$ is disjoint to all events
	\item if $a \cap B = \emptyset$ then $A \subset B^c$
	and $B \subset A^c$
	\item if $A \subset B$, then $B^c \subset A^c$
	\item $A \setminus B = A \cap B^c$ and $B \setminus A = A^c \cap B$
	\item $A \triangle B = (A \cup B) \setminus (A \cap B) $
\end{enumerate}

%
%End of lecture 2
%

\begin{remark}
	Consider $\textbf{S} = \{H, T\}$ and $\mathscr{P}(\textbf{S}) = \{\emptyset, 
	\{H\}, \{T\}, \textbf{S} \}$ \\
	\indent If \textbf{S} is countable, we can take $\mathscr{P}(\textbf{S}) $
	as the domain for probability function P.  \\
	\indent However, if \textbf{S} is uncountable, then \textbf{S} is too large, 
	and it is not possible to define a function for $\mathscr{P}(\textbf{S}) $
\end{remark}

\begin{remark}
\textbf{Desired Properties of P(.)}
\begin{enumerate}
	\item $P(\emptyset) = 0$
	\item If $A$ and $B$ are disjoint then $P(A \cup B) = P(A) + P(B)$
	\item $P(A^c) = 1- P(A)$
	\item If $A \subset B$ then $P(A) \leq P(B)$
	\item $P(A) \leq 1$
\end{enumerate}
\end{remark}

%
%End Lecture 3
%

\begin{remark}
	Partition $\{E_1, \: ..., \: E_n \}$ is a finite partition
\end{remark}

\begin{remark}
	For \textbf{S}, $\{A, A^c\}$ is a partition
\end{remark}

\begin{remark}
	If $E_n: n \geq 1$ is a partition of $A$, then $P(A) = \sum_{i=1}^{\infty} P(E_i)$
\end{remark}

\begin{enumerate}
	\item Suppose $A$ and $B$ are two events.  Then $\{A \cap B, A^c \cap B\}$
	is a partition of $B$.  Also, $P(A^c \cap B) = P(B) - P(A \cap B)$ 

	\begin{myproof}
		$(A \cap B) \cap (A^c \cap B) = 
		(A \cap A^c) \cap (B \cap B) = 0$  \textit{(pairwise disjoint)} \\
		$(A \cap B) \cup (A^c \cap B) = (A \cup A^c) \cap B = B $ 
		\textit{(exhaustive)} 
	\end{myproof}

	\item $A$ and $B$ are two events.  $P(A \cup B) = P(A) + P(B) - P(A \cap B)$ \\
	(This is the generalized version of b)) 

	\item if $\{C_1, C_2, ... \}$ is a partition of \textbf{S}, then \\
	$P(A) = \sum_{i=1}^{\infty} P(A \cap C_i)$ 

	\item\textbf{Boole's Inequality} \\
	\indent $P(\cup_{i=1}^{\infty} A_i) \leq \sum_{i=1}^{\infty}P(A_i)$ 
	
	\item \textbf{Bonferroni's Inequality}\\
	$P(\cap_{i=1}^{\infty} A_i) \geq 1 - \sum_{i=1}^{\infty} P(A_i^c)$
\end{enumerate}

\begin{remark}
	Proving Bonferroni from Boole: \\
	$P(\cup_{i=1}^{\infty} A_i^c) 
	\leq \sum_{i=1}^{\infty} P(A_i^c)$ \\
	$1- P(\cup_{i=1}^{\infty} A_i^c) \:  
	\geq 1 - \sum_{i=1}^{\infty} P(A_i^c)$ \\
	$ = P[(\cup_{i=1}^{\infty} A_i^c)^c] \: 
	\geq$ ...  \\
	$ = P[\cap_{i=1}^{\infty} (A_i^c)^c] \: 
	\geq$ ... \\
	$ = P(\cap_{i=1}^{\infty} A_i ) \: 
	\geq 1- \sum_{i=1}^{\infty} P(A_i^c)$ 
\end{remark}

\begin{enumerate}
	\item If $A_n \uparrow A$, then $P(A) = \lim_{n \to \infty} P(A_n)$ 

	\item If $B_n \downarrow B$ then $P(B) = \lim_{n \to \infty} P(B_n)$ 
\end{enumerate}

%
%End Lecture 4
%

\begin{remark}
	Suppose for some set $A, P(A) = 1$ Does this imply $A=S$? \\
	Does $P(B) = 0$ imply $B \neq \emptyset$? \\
	Does $P(A \cap B) = 0$ imply $A,B$ are disjoint? \\
	Not necessarily for all above! 
\end{remark}

%
%End Lecture 5
%







\end{document}























